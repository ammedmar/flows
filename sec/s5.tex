% !TEX root = flows.tex

\section{Flow comparison theorem}\label{S:flow comparison theorem}

\subsection{Discussion}

We now come to the main result of this paper.

\maintheorem*

The time needed to flow to obtain the equality between intersection and cup product will vary depending on $W$ and $V$, but on any finite subcomplex we have the following uniformity.

\begin{corollary}\label{C:finite diagram}
	Let $M$ be a closed manifold with smooth cubulation $|X| \to M$, and let $F^*$ be a finitely-generated chain complex with chain map $g \colon F^*\to C^*_{\Gamma \pf}(M)$.
	Then, there is a $T \in \R$ such that for all $t > T$ the following diagram commutes:
	\begin{equation*}
		\begin{tikzcd} [row sep=tiny]
			& C^*_{\Gamma \pf}(M)^{\otimes 2} \arrow[r, "\cI \otimes \cI"] & \cochains(X)^{\otimes 2} \arrow[dd, "\sms"] \\
			F^*\otimes F^* \arrow[ur, in=180, out=45,"g\otimes g"] \arrow[dr, in=180, out=-45, "\f_t \circ g \; \uplus \; \f_{-t} \circ g"']& & \\
			& C^*_{\Gamma \pf}(M) \arrow[r, "\cI"] & \cochains(X).
		\end{tikzcd}
	\end{equation*}
\end{corollary}

In particular, if $F^*$ is the subcomplex of $C^*_\Gamma(M)$ generated by two cochains $\uW$ and $\uV$, this says that the diagram commutes for large enough $t$ starting with the chain $\uW\otimes\uV$ on the left, recapitulating \cref{T:main theorem} as a statement about cochains, not just elements of $\cman_\pf^*(M)$.
There are several other candidates for a useful finitely-generated complex $F^*$, with the most desirable being those whose maps to $C^*_{\Gamma \pf}(M)$ induce quasi-isomorphisms.
In that case, \cref{C:finite diagram} shows that the finitely-generated complex provides a geometric model for cubical cochains, though only as a differential graded associative algebra.
As discussed in the Introduction, we plan to strengthen this connection beyond the associative setting in future work.

The ideal example of a finitely-generated subcomplex of geometric cochains that is quasi-isomorphic to cubical cochains through intersection
would be generated by a dual complex to the cubulation whose cells are smooth manifolds with corners.
Unfortunately, it is not clear that such dual complexes always exist except in special cases, for example on two-dimensional manifolds.

Another example we can give for a useful $F^*$ comes from considering an oriented manifold $M$ with a smooth triangulation $|K|\to M$ by a finite simplicial complex $K$ with a vertex ordering.
Using the orientation $\beta_M$ of $M$, every simplicial face inclusion $\sigma\to M$ with $\sigma$ a simplex of $K$ is co-oriented by $(\beta_\sigma,\beta_M)$, where $\beta_\sigma$ is the standard orientation that arises from the ordering of the vertices of $\sigma$.
Thus we have an inclusion homomorphism $C_*(K)\into C^{m-*}_\Gamma(M)$, where $m = \dim(M)$.
This inclusion is a quasi-isomorphism, as the inclusion of $C_*(K)$ into the smooth singular chain complex $C_*^{ssing}(M)$ of $M$ is a known quasi-isomorphism, the map $C_*^{ssing}(M) \to C_*^\Gamma(M)$ is observed to be a quasi-isomorphism in \cite[Section 10]{Lipy14}, and $C_*^\Gamma(M)=C^{m-*}_\Gamma(M)$ for $M$ closed and oriented (see \cite[Section 12]{Lipy14} or \cite[Theorem 4.21]{medina2022foundations}).
We may then modify the triangulation $|K|\to M$ by postcomposing with a map $f \colon M\to M$ that is homotopic to the identity but which shifts each simplex of $K$ into general position with respect to every face inclusion of the cubulation $|X|\to M$.
This can be done as in the proof of \cref{T:transverse complex}, as given in \cite[Theorem 6.5]{medina2022foundations}, using the techniques of \cite[Section 2.3]{GuPo74}.
As $f$ is homotopic to the identity, we obtain the composite quasi-isomorphism $C_*(K)\into C^{m-*}_\Gamma(M) \xr{f} C^{m-*}_\Gamma(M)$, but now with image of the composite in $C^{m-*}_{\Gamma \pf X}(M)$.
So by the following diagram,
\begin{equation*}
	\begin{tikzcd}[column sep=tiny, row sep=tiny]
		C_*(K) \arrow[rr, "f"] \arrow[dr, out=-90, in=180]& & C^{m-*}_{\Gamma}(M) \\
		& C^{m-*}_{\Gamma \pf X}(M) \arrow[ur, out=0, in =-90] &
	\end{tikzcd}
\end{equation*}
we obtain a quasi-isomorphism from a finitely generated chain complex $C_*(K) \to C^{m-*}_{\Gamma \pf X}(M)$ to which our results can be applied with a single bound.

\bigskip

The proof of \cref{T:main theorem} is contained in the following sections.
As the intersection homomorphism is defined through evaluation on cubes in $X$, the proof is ultimately ``local," proceeding over isolated cubes.
We thus make the following definition that will be useful throughout, with $E$ being a cube our main case of interest.

\begin{definition}
	Let $M$ be a manifold without boundary and $E$ and $W$ transverse elements of $\cman^*_\Gamma(M)$ with $E$ embedded.
	We denote the pull-back $W \times_M E$ by $W_E$.
	By \cref{pullback}, $W_E$ is a c-manifold over $E$, and by \cite[Sections 3.5.1 and 3.5.2]{medina2022foundations} the map $W_E\to E$ can be endowed with a pull-back co-orientation induced by the co-orientation of $W\to M$ that does not depend on the orientation of $E$.
\end{definition}

\subsection{Transversality}

We begin with the first claim of \cref{T:main theorem}, of transversality.
The version of transversality needed for all further claims
is not just over the entire manifold but also for restrictions over any cube.

\begin{theorem}\label{T:transversality}
	Let $|X| \to M$ be a cubulation of a manifold without boundary.
	Suppose $W$ and $V$ are in $\cman_{\Gamma \pf}^*(M)$.
	Then, for any cube $E$ in $X$, both $\f_t(W)_E$ and $\f_{-t}(W)_E$ are transverse to $V_E$ for $t$ sufficiently large.
\end{theorem}

\begin{proof}
	Let $\dim(E) = n$, $\dim(W) = w$, and $\dim(V) = v$.
	Let us identify $E$ with $\interval^n$ and omit it from notation, so $\f_t(W)_E$ and $V_E$ become $\f_t(W)$ and $V$.
	It is sufficient to prove the statement for the top stratum of $W$ and $V$, as the argument will apply to any deeper strata.
	We will establish the following statement inducting over $i = 0, \dots, n$.
	\begin{itemize}
		\item[($\ast$)] There exists a neighborhood $\mathcal N^i$ of $\term_{i}(\interval^n)$ and a $T^i \in \R$ such that $\f_t(W)$ is transverse to $V$ within $\mathcal N^i$ for all $t > T^i$.
	\end{itemize}
	Since $\interval^n = \term_{n}(\interval^n)$, this will suffice.

	For the base case of the induction we consider $\term_0(\interval^n) = \{\underline{1}\}$.
	If $\dim(V) < n$ then, by the assumption that $V$ is transverse to $\interval^n$, there is a neighborhood of $\underline{1}$ that does not intersect $V$.
	In this case ($\ast$) is fulfilled vacuously.
	If $\dim(V) = n$, then again the condition is fulfilled vacuously if $V$ does not contain $\underline{1}$, so we assume $\underline{1} \in V$.
	Because of transversality, the Inverse Function Theorem implies that $V \to \interval^n$ is a local diffeomorphism onto a neighborhood $\mathcal N^0$ of $\underline{1}$ and, therefore, it is transverse to any map therein, so we can take $T^0 = 0$.

	Let us establish now the induction step going from $i-1$ to $i$.
	Consider $F \in \term_{i}(\interval^n)$ and notice that the union of $\beta_F = \{\e_i\ |\ i \in F_{01}\}$ and $\beta_{F^-} = \{\e_i\ |\ i \in F^-_{01}\}$ trivialize the tangent bundle of $\interval^n$.

	Choose $\delta > 0$ small enough so that the closed $L^\infty$ neighborhood $\overline N_\delta(F)$ of $F$ is such that for any $y \in V \cap \overline N_\delta(F)$ the affine space $T_yV$ is transverse to the span of $\beta_F = \{\e_i\ |\ i \in F_{01}\}$ at $y$.

	Similarly, consider a $\zeta > 0$ small enough so that the closed $L^\infty$ neighborhood $\overline N_\zeta(F^-)$ of $F^-$ is such that either $W \cap \overline N_\zeta(F^-) = \emptyset$ or for any $x \in W \cap \overline N_\zeta(F^-)$ the affine space $T_xW$ is transverse to the span of $\beta_{F^-} = \{\e_i\ |\ i \in F^-_{01}\}$ at $x$.
	In the latter case, for every such $x$, the transversality implies that $T_xW$ projects surjectively onto $F$; in other words, the projection of $W \cap \overline N_\zeta(F^-)$ to $F$ is a submersion onto its image.
	Therefore, we may choose continuously with $x$ a subset of $T_xW$ of the form $\beta_x = \{\e_j + v_j\ |\ j \in F_{01}\}$ such that $v_j$ is in the span of $\beta_{F^-}$.
	As $W \cap \overline N_\zeta(F^-)$ is compact, the $v_j$ have bounded norm.

	By possibly making $\delta$ smaller we can choose $u \in (0, 1)$ by inductive hypothesis such that $\overline N_\delta(F) \setminus \overline N_\delta L_u(F)$ is contained in $\mathcal N^{i-1}$.

	By \cref{L:flow to initial and terminal faces} we may choose $t > T^{i-1}$ sufficiently large so that all points in $\f_t(W) \cap V \cap \overline N_\delta L_u(F)$ are of the form $y = \f_t(x)$ for some $x \in W \cap \overline N_\zeta(F^-)$.

	We use \cref{L:jacobian ratios} to deduce that the push forward of the span of $\beta_x$ along $D\,\f_t$ is as close as desired to the span of $\beta_F$ at $y$ and is therefore transverse to $T_y V$.
	The induction step is completed by taking $\mathcal N^i$ to be the union over all terminal faces of $N_\delta(F)$ and $T^i$ the maximum value of their associated $t$.
\end{proof}

If $M$ is closed, then any cubulation of $M$ is by a finite number of cubes.
Applying this theorem for each top-dimensional cube yields that if $W$ and $V$ are in $\cman_{\Gamma \pf}^*(M)$ then both $\f_t(W)$ and $\f_{-t}(W)$ are transverse to $V$ for $t$ sufficiently large.
We have the following consequence of this theorem related to \cref{T:main theorem}.

\begin{corollary}\label{C:transversality}
	Let $M$ be a cubulated closed manifold.
	If $W$ and $V$ are in $\cman_{\Gamma \pf}^*(M)$ then for $t$ sufficiently large $\f_t(W)$ and $\f_{-t}(V)$, as well $\f_{-t}(W)$ and $\f_t(V)$, are transverse over $M$.
\end{corollary}

\begin{proof}
	By \cref{T:transversality} there is a $t$ large enough so that $\f_{2t}(W)$ and $V$ are transverse for all larger $t$.
	Now apply $\f_{-t}$ to both terms and that transversality is preserved by composition with a diffeomorphism.
	Similarly, we can apply $\f_t$ to $\f_{-2t}(W)$ and $V$ for large enough $t$.
	Now choose $t$ large enough to do both.
\end{proof}

\subsection{Locality}\label{S: locality}

On a cubulated manifold the logistic vector field is compatibly defined across cubes by \cref{L:f is natural}.
In this subsection we show that we can analyze the pull-back product over $M$ of cochains $\f_t(W)$ and $V$ locally -- that is, cube-by-cube.

\begin{lemma}\label{L:little trans lemma}
	Let $V$ and $W$ be c-manifolds over $M$ and $S$ a submanifold without boundary of $M$.
	Suppose $V$, $W$, and $S$ are pairwise transverse.
	If $V_S$ and $W_S$ are transverse over $S$ then $W \times_M V$ is transverse to $S$ over $M$.
	Moreover,	$W_S \times_S V_S \cong (W \times_M V) \times_M S$.
\end{lemma}

We will later apply this when $S$ is the interior of a cube in a cubical structure.

\begin{proof}
	The last statement of the lemma follows from identifying both $W_S \times_S V_S$ and $ (W \times_M V) \times_M S$ with
	the subspace of triples of $(x,y,z) \in W \times V \times S$ such that $r_W(x) = r_V(y) = r_S(z)$.

	This observation at the level of tangent spaces	also gives rise to the first statement, recalling that the tangent bundle of the fiber product is the fiber product of the tangent bundles.
	Indeed, transversality of two maps at a point where they coincide is defined locally by surjectivity of the map $(\vec a,\vec b)\to D_xr_W(\vec a)-D_yr_V(\vec b)$ from the direct sum of tangent spaces of the domain points to that of the ambient manifold.
	By exactness, this surjectivity is equivalent to the kernel having the appropriate dimension, but the kernel is precisely the tangent space of the	fiber product.
	Thus, it follows from a short computation that two maps are transverse if and only if the the fiber-product of tangent spaces over any point has codimension equal to the sum of the codimensions of the tangent spaces.
	As in the ``global" case, the fiber product of $T_{(x,z)} V_S$ and $T_{(y,z)} W_S$ over $T_zS$ coincides with that of $T_{(x,y)} (W \times_M V)$ and $T_z S$ over $T_z M$, as both are identified with the same subspace of $T_x W \times T_y V \times T_z S$.
	But the first pull-back is transverse by assumption, so this subspace has codimension equal to the sum of the codimensions of $S$, $V$, and $W$ in $M$.
	This codimension is also what is required for the second pull-back to be transverse, yielding the result.
\end{proof}

\begin{proposition}\label{P:locality}
	Let $M$ be a cubulated closed manifold.
	Suppose $W$ and $V$ are in $\cman_{\Gamma \pf}^*(M)$.
	For $t$ sufficiently large, the pull-back $\f_t(W) \times_M V$ is transverse to the cubulation, and
	\begin{equation*}
		\f_t(W) \times_M V = \bigcup_{E \in X} {\f_t(W_E)} \times_E V_E
	\end{equation*}
	as spaces, with each ${\f_t(W_E)} \times_E V_E$ a manifold with corners.
\end{proposition}

\begin{proof}
	By \cref{L:flow preserves transversality}, the c-manifold $\f_t(W)$ remains transverse to the cubulation for every $t$.
	By \cref{T:transversality}, over each $E$ in the cubulation, $\f_t(W)_E$ is transverse to $V_E$ for $t$ sufficiently large.
	Taking the largest
	such $t$, we can then apply \cref{L:little trans lemma} to obtain transversality of the pull-back of $\f_t(W)$ and $V$ to the entire cubulation.

	Tautologically, $ \f_t(W) \times_M V = \bigcup_{E \in X} {\f_t(W)}_E \times_E V_E$.
	But again using that the logistic flow respects the cubulation, ${\f_t(W)}_E = {\f_t(W_E)}$, which gives the decomposition.
	That this is itself a manifold with corners follows from the identification by \cref{L:little trans lemma} of $ {\f_t(W_E)} \times_E V_E$ with $(\f_t(W) \times_M V)\times_M E$, the latter being a manifold with corners by our transversality result.
\end{proof}

For the following lemma, recall \cref{D:intersection number}.

\begin{lemma}\label{L:intersection signs}
	Let $W$ and $V$ be co-oriented c-manifolds over a closed manifold $M$ that are transverse to a cubulation $|X| \to M$, and let $E$ be a cube whose dimension
	is the sum of the codimensions of $W$ and $V$ in $M$.
	Then, for sufficiently large $t$, $\f_t(W_E)$ and $V_E$ are complementary in $E$ and the value of $\cI(\f_t(W) \times_M V)$ on $E$
	is equal to $I_E(\f_t(W_E), V_E)$.
\end{lemma}

\begin{proof}
	We assume that $t$ is as large as needed for \cref{P:locality}.
	In this case, both $(r_{\f_t(W) \times_M V})^{-1}(E)$ and $\rm{Int}_E(\f_t(W_E), V_E)$ consist of points $(x,y,z) \in W \times V \times E$ with $(\f_t \circ r_W)(x) = r_V (y) = \iota_E(z)$.
	We check that this is an isomorphism of signed sets.

	Consider one such intersection point.
	We first compute its contribution to $I_E(\f_t(W_E), V_E)$.
	As $\f_t(W)_E$ and $V_E$ are transverse and of complementary dimension in $E$, these spaces are embedded in $E$ near the intersection point.
	Moreover, $\f_t W$ and $V$ are embedded in $M$ near such a point, since a nontrivial $\vec v$ in the kernel of the derivative of $r_{\f_t (W)}$ would imply $(\vec v,0)\in TW\times_{TM} TE$, and it maps to $0$ in $T_M$, producing a nontrivial kernel for the derivative of $r_{\f_t(W_E)}=r_{\f_t(W)_E}$.
	Similarly for $V$ and $V_E$.
	The codimensions of $\f_t (W_E)$ and $V_E$ in $E$ agree with the codimensions of $W$ and $V$ in $M$, and their normal bundles in $E$ are the restriction of the normal bundles of $\f_t(W)$ and $V$ in $M$ where they are embedded.
	Thus we can use normal co-orientations and identify the normal co-orientation $\beta_{\f_t(W)}$ of $\f_t(W)_E$ in $E$ with the normal co-orientation of $\f_t(W)$ in $M$ and similarly for $V_E$ and $V$.
	Then the sign of the intersection is $1$ if $\beta_W \wedge \beta_V$ agrees with the orientation of $E$ and is $-1$ otherwise.

	Now consider the pullback $(r_{\f_t(W) \times_M V})^{-1}(E)$.
	As $\f_t( W)$ and $V$ are embedded near $x$ and $y$, we co-orient $\f_t(W) \times_M V$ at $z$ again by $\beta_{\f_t(W)} \wedge \beta_V$ according with the properties of co-orientations of fiber products of embeddings; see \cref{T:pull-back co-or}.
	But now again the contribution to $\cI(\f_t(W) \times_M V)$ is $+1$ or $-1$ as this pair agrees or not with the orientation of $E$.
\end{proof}

\begin{corollary}\label{C:intersection signs}
	With the assumptions of \cref{L:intersection signs}, for sufficiently large $t$, $\f_t(W_E)$ and $\f_{-t}(V_E)$ are complementary in $E$ and the value of $\cI(\f_t(W) \times_M \f_{-t}(V))$ on $E$ is equal to $I_E(\f_t(W_E), \f_{-t}(V_E))$.
\end{corollary}

\begin{proof}
	By the preceding lemma, there is a $t$ large enough that $\f_{2t}(W_E)$ and $V_E$ are complementary in $E$ and the value of $\cI(\f_{2t}(W) \times_M V)$ on $E$ is equal to $I_E(\f_{2t}(W_E), V_E)$.
	Now apply the diffeomorphism $\f_{-t}$ to $\f_{2t}(W_E)$ and $V_E$.
	As $\f_{-t}$ preserves orientations and co-orientations, the corollary follows.
\end{proof}

\subsection{Graph-like neighborhoods}

We now focus on the local structure of a transverse c-manifold over $\interval^n$ around intersection points with faces of complementary dimension.
The Implicit Function Theorem guarantees that the following subspaces occur naturally in this setting.

\begin{figure}[!h]
	\centering
	\begin{subfigure}{.32\textwidth}
		\includegraphics[scale=.7]{figures/graph_like_ngbhd2.pdf}
		\hfill
	\end{subfigure}
	\begin{subfigure}{.32\textwidth}
		\hfill
		\includegraphics[scale=.7]{figures/graph_like_ngbhd.pdf}
	\end{subfigure}
	\caption{Examples of graph-like neighborhoods with $p$ in the interior of faces $F=(\emptyset, \{2,3\}, \{1\})$ and $F=(\{2\}, \{3\}, \{1\})$, respectively.
		In the diagram on the right, $F^-\times F^+$ is difficult to depict so we draw the domain $D$ in the bottom face of the cube, which is also complementary to $F$.}
	\label{F:graph like neighborhood}
\end{figure}

\begin{definition}\label{D:graph-like}
	Let $F$ be a face of $\interval^n$ and $p$ a point in its interior.
	Consider the $F$-decomposition $\interval^n \cong F^- \times F \times F^+$ and its canonical projections $\pi_F^{\perp}$ to $F^- \times F^+$ and $\pi_F$ to $F$.
	A set $C$ is said to be a \textbf{graph-like neighborhood} of $p$ if
	\begin{enumerate}
		\item $C$ is the graph of a smooth map $s \colon D \to F$, where $D$ is an open neighborhood of $\pi_F^\perp(p)$ in $F^- \times F^+$ such that $D$ intersects only the faces of $F^- \times F^+$ containing $\pi_F^\perp(p)$ and is bounded away from the other faces of $F^- \times F^+$,
		\item $s(\pi_F^\perp(p)) = p$,
		\item $C$ is bounded away from the faces of $F$ that do not contain $p$, and
		\item $C$ is transverse to all faces of $\interval^n$,
	\end{enumerate}
	where, as usual, two sets are said to be bounded away from each other if there is a positive minimum distance between every point in one and every point in the other.
\end{definition}

In general $C$ is not actually a neighborhood of $p$ in $\interval^n$ as $\dim(C)=n-\dim F$, with an extreme case being $F =\interval^n$ in which case $C$ is just a point in the interior of $F$.
For any graph-like neighborhood $C$ of a point in the interior of a face $F$, there exists by (3) $r, u \in (0,1)$ such that $C \subseteq N_rL_u(F)$ and $C \subseteq N_rU_u(F)$.

Since the logistic flow can be expressed independently in each coordinate, it preserves the property of being a graph-like neighborhood.

Let $W$ be a c-manifold over $\interval^n$.
Assume that for any $(n-\dim W)$-face $F$ of $\interval^n$, the set $W \cap F$ consists of a finite number of points near which $W$ is locally embedded.
By the Implicit Function Theorem, the discrete set $W \times_{\interval^n} F$ can be used to parameterize a collection of graph-like neighborhoods of the points in $W \cap F$.
Recall the definition of reciprocal faces of $\interval^n$, \cref{D:reciprocal}.

\begin{lemma}\label{L:flow intersection for graph-like nbhds}
	Let $F$ and $F^\prime$ faces of $\interval^n$ of complementary dimension and let $C$ and $C^\prime$ be graph-like neighborhoods of points in their respective interiors.
	If the pair $(F,F^\prime)$ is reciprocal, then for $t$ sufficiently large $\f_t(C) \cap \f_{-t}(C^\prime)$ is a single point.
	If the pair is not reciprocal then for $t$ sufficiently large this intersection is empty.
\end{lemma}

\begin{figure}[!h]
	\includegraphics[scale=.65]{figures/symmetric}
	\caption{Proof of \cref{L:flow intersection for graph-like nbhds}.}
	\label{F:intersection}
\end{figure}

\begin{proof}
	We refer the reader to \cref{F:intersection} to accompany the proof.

	First suppose the pair $(F,F^\prime)$ is reciprocal.
	This is the case if and only if there is a vertex $v$ with $v^- = F$ and $v^+ = F^\prime$.
	Let $s^\pm \colon D^\pm \subseteq v^\pm \to \interval^n = v^- \times v^+$ be sections of $\pi_{v^\pm}^\perp$ such that $C = s^+(D^+)$ and $C^\prime = s^-(D^-)$.
	Let us consider a closed $L^\infty$ $\epsilon$-neighborhood $\overline N_\epsilon(v)$ of $v$ in $\interval^n$ with $\epsilon$ small enough so that $N_\epsilon(v^\pm) \subset D^\pm$.
	Write $\overline N_\epsilon^\pm = \overline N_\epsilon(v) \cap v^\pm$.
	Let us assume $t$ large enough so that $\norm{\f_{\pm t}(s^\pm(v)) - v} < \epsilon$, which is possible because $s^\pm(v)$ are in the interiors of $v^\pm$ by assumption and $v$ is the terminal vertex of $v^-$ and the initial vertex of $v^+$.
	Let $\overline N_t^\pm = \overline N_\epsilon(v) \cap \big(\f_{\pm t}\circ s^\pm (D^\pm)\big)$.
	As $\f_t$ flows each coordinate independently, $\f_t(C)$ remains a graph of $\f_t(D)$ for all $t$.
	Therefore, by Lemmas \ref{L:flow to initial and terminal faces}, \ref{L:jacobian ratios}, and \ref{L:domain flow}, for large enough $t$ the set $\overline N_t^\pm$ will be as small a perturbation as desired of $\overline N_\epsilon^\pm$ and additionally the tangent space at each point of $\overline N_t^\pm$ will be as small a perturbation as desired of the ``plane" containing $v^\pm$.
	Therefore, since $\overline N_\epsilon^+ \cap \overline N_\epsilon^- = \{v\}$, the stability of transverse intersections implies that there exists a unique $y_t \in \overline N_t^+ \cap \overline N_t^-$.
	The lemma follows in this case since there cannot be intersections outside $\overline N_\epsilon(v) = \overline N_\epsilon(v^+) \cap \overline N_\epsilon(v^-)$ since, by \cref{L:flow to initial and terminal faces}, we can assume $t$ large enough so that $\f_t(C) \subseteq \overline N_\epsilon(v^+)$ and $\f_{-t}(C^\prime) \subseteq \overline N_\epsilon(v^-)$.

	Next we assume that the pair $(F,F^\prime)$ is not reciprocal (but still of complementary dimension).
	In particular, $F$ is not initial or $F^\prime$ is not terminal.
	Let $r,u,r^\prime,u^\prime \in (0,1)$ such that $C \subseteq N_rU_u(F)$ and $C^\prime \subseteq N_{r^\prime}L_{u^\prime}(F^\prime)$, which is possible by the definition of graph-like neighborhoods.
	First suppose $F$ is not initial.
	By \cref{L:flow to initial and terminal faces} we have that for any $\epsilon > 0$ and all $t$ sufficiently large $\f_{2t}(C) \subseteq N_\epsilon(F^+)$.
	As $F$ is not initial, $\dim F^+ < n-\dim F$, so for large enough $t$ we have $\f_{2t}(C)$ contained in a neighborhood of the $n-\dim F-1$ skeleton of $\interval^n$.
	But now $\dim(C^\prime) = n-\dim F^\prime = \dim F$, and $C^\prime$ is supposed by definition to be transverse to $\interval^n$ and bounded away from the faces of $\interval^n$ it does not intersect.
	Thus for sufficiently large $t$ we have $\f_{2t}(C)\cap C^\prime = \emptyset$, and as $\f_{-t}$ is a diffeomorphism, applying it to both terms we obtain $\f_{t}(C) \cap \f_{-t} (C^\prime) = \emptyset$.
	The argument if $F^\prime$ is not terminal is analogous.
\end{proof}

The preceding lemma showed that for large enough $t$ the intersection $\f_t(C) \cap \f_{-t}(C^\prime)$ is empty unless the pair $(F,F^\prime)$ is reciprocal, in which case the intersection is a single point.
The next lemma determines how the sign of that intersection point, when it exists, depends on the co-orientations of $C$ and $C^\prime$.

Recall that the faces of $\interval^n$ have a canonical orientation induced from the order of the basis $\{\e_1, \dots \e_n\}$.
Therefore, a \textbf{compatible co-orientation} can be defined for any graph-like neighborhood $C$ to be the normal co-orientation induced from the orientation of the face $F$ in complementary dimensions that $C$ intersects.
In other words, the compatible co-orientation of $C$ is determined by $\cI(C)(F) = +1$.
Note that $C$ is diffeomorphic to its domain $D$ and so it is a manifold with corners.
In general it will not be properly embedded so we abuse notation in writing $\cI(C)$, but the meaning should remain clear.

Recall that for $v \in \vertices(\interval^n)$ the shuffle sign $\sh(v)$ is the sign of the shuffle permutation of $\{1, \dots, n\}$ placing the ordered free variables in $v^-$ before those in $v^+$.

\begin{lemma}\label{L:sign for graph-like nbhds intersection}
	Let $v \in \vertices(\interval^n)$.
	If $C$ and $C^\prime$ are compatibly co-oriented graph-like neighborhoods of points respectively in the interiors of $v^-$ and $v^+$, then, for $t$ sufficiently large,
	\begin{equation}
		\label{E:sign flow intersection graph-like 1}
		\cI \big( \f_t(C) \times_{\interval^n} \f_{-t}(C^\prime) \big)
		\big( [\underline{0}, \underline{1}]^{\otimes n} \big) = \sh(v).\\
	\end{equation}
\end{lemma}

\begin{proof}
	We start by noticing that the hypotheses imply that $C$ and $C'$ have complementary dimensions in $\interval^n$ since $v^-$ and $v^+$ do.

	By \cref{L:flow intersection for graph-like nbhds}, the fiber product $\f_t(C) \times_{\interval^n} \f_{-t}(C^\prime)$ is a single point.
	Since the logistic flow is an orientation-preserving diffeomorphism, it preserves co-orientation as well.
	In terms of ordered bases, by \cref{T:pull-back co-or} the co-orientation of the fiber product is determined by the orientation of the normal bundle of the intersection point determined by the ordered concatenation $\beta_- \cup \beta_+$, where $\beta_\pm$ is an ordered basis representing the orientation of $v^\pm$.
	From this the claim follows.
\end{proof}

\subsection{The proof of \cref{T:main theorem}}

In this section we prove our main theorem; let us first recall its statement.
Let $M$ be a cubulated closed manifold.
For $W, V \in \cman_\pf(M)$ and $t$ sufficiently large:
\begin{enumerate}
	\item $\f_t(W)$ and $\f_{-t}(V)$ are transverse and
	\begin{equation*}
		\cI \big( \f_t(W) \times_M \f_{-t}(V) \big) =
		\cI \big( \f_t(W) \big) \sms \cI \big( \f_{-t}(V) \big).
	\end{equation*}
	\item $\f_{-t}(W)$ and $\f_t(V)$ are transverse and
	\begin{equation*}
		\cI \big( \f_{-t}(W) \times_M \f_t(V) \big) =
		(-1)^{|W||V|} \, \cI \big( \f_t(V) \big) \sms \cI \big( \f_{-t}(W) \big),
	\end{equation*}
	where $|W||V|$ is the product of the codimensions of $W$ and $V$ over $M$.
\end{enumerate}

The transversality statement was proven as \cref{T:transversality}.
By \cref{P:locality} and \cref{C:intersection signs}, it suffices to consider pull-backs over an arbitrary $n$-face, where $n$ is the sum of the codimensions of $V$ and $W$ over $M$.
We identify this face with $\interval^n$ and, by abuse, denote the pull-backs of $V$ and $W$ over this face also by $V$ and $W$.

Let us consider (1) first.
Since $W$ is transverse to all faces of $\interval^n$ in particular we have that its intersection with faces of complementary dimension are discrete.
By the Implicit Function Theorem, for any such $F$ there are local neighborhoods in $W$ of the points in $W \cap F$ which are graph-like.
Furthermore, the pull-back $W \times_{\interval^n} F$ can be used to parameterize these graph-like neighborhoods, which we assume equipped with the co-orientation induced from $W$.
We use this co-orientation to endow $W \times_{\interval^n} F$ with a sign function,
sending an element to $+1$ if the orientation of the normal bundle to the graph-like neighborhood given by its co-orientation agrees
with the orientation of $F$ and to $-1$ if not.
A similar situation applies when considering $V$ and faces of dimension $\dim W$.

Let $\Ginit_W$ (resp. $\Gterm_V$) be the union of the parameterizing sets of these neighborhoods over initial (resp. terminal) faces.
Explicitly,
\begin{equation*}
	\Ginit_W\ =
	\bigcup_{F \in \init_{\dim V}(\interval^n)} W \times_{\interval^n} F
	\qquad \text{ and } \qquad
	\Gterm_W\ =
	\bigcup_{F^\prime \in \term_{\dim W}(\interval^n)} V \times_{\interval^n} F^\prime.
\end{equation*}
Moreover, let
\begin{equation*}
	Z_W^\init =\, W\ \setminus \bigcup_{p \in \Ginit_W} C_p
	\qquad \text{ and } \qquad
	Z_V^\term =\, V\ \setminus \bigcup_{p \in \Gterm_V} C_p^\prime.
\end{equation*}

If $\dim V=n$, then $Z_W^\init=\emptyset$, and similarly if $\dim V=0$ then $Z_V^\term=\emptyset$.
So we focus on $\dim V<n$.
We notice that $Z_W^\init$ is contained in the complement of a neighborhood $\init_{\dim V}(\interval^n)$.
Therefore, no point in $Z_W^\init$ can have $n-\dim V$ (or more) coordinates equal to $0$, and in fact we can assume there is an $\eta>0$
so that every point of $Z_W^\init$ has more than $\dim V$ coordinates that are larger than $\eta$.
So, for any $\epsilon>0$ we can choose a large enough $t$ to ensure that every point of $\f_t(Z_W^\init)$ has more than $\dim V$ coordinates greater than
$1-\epsilon$.
In other words, $\f_t(Z_W^\init)$ can be contained in any given neighborhood of $\term_{n-\dim V-1}(\interval^n)$.

As $V$ maps properly and transversely to $\interval^n$, there is a neighborhood of the $n-\dim(V)-1$ skeleton of $\interval^n$ that is disjoint from $V$, and so we have shown that $\f_t(Z_W^\init) \cap V = \emptyset$ for large enough $t$.
Replacing $t$ with $2t$ in the preceding sentence and then applying $\f_{-t}$ to $\f_{2t}(Z_W^\init)$ and $V$ we obtain that $\f_t(Z_W^\init) \cap \f_{-t}(V) = \emptyset$ for large enough $t$.
Similarly, we can find $t$ large enough so that $\f_{t}(W) \cap \f_{-t}(Z_V^\term) = \emptyset$.
This shows that
\begin{equation}\label{E:reduction to C}
	\f_t(W) \times_{\interval^n} \f_{-t}(V) = \f_t \left( \bigcup_{p \in \Ginit_W} C_p \right) \times_{\interval^n} \f_{-t} \left( \bigcup_{p \in \Gterm_V} C_p^\prime \right).
\end{equation}

\cref{L:flow intersection for graph-like nbhds} now implies that for $t$ large enough the discrete set \eqref{E:reduction to C} is in bijection with the discrete set
\begin{equation*}
	S\ \ =\!\! \bigcup_{v \in \vertices(\interval^n)} \left(W \times_{\interval^n} v^-\right) \times \left(V \times_{\interval^n} v^+\right) \ \subseteq\ \Ginit_W \times \Gterm_V,
\end{equation*}
where we are running over vertices so that the sum of the dimensions of $v^-$ and $v^+$ is $n$.
Considering $\left(W \times_{\interval^n} v^-\right)$ and $\left(V \times_{\interval^n} v^+\right)$ as signed sets as just above, we define a sign function on $S$ by sending a pair $(\xi, \eta) \in S$ to the product of the sign of $\xi \in \left(W \times_{\interval^n} v^-\right)$, the sign of $\eta \in \left(V \times_{\interval^n} v^+\right)$, and the shuffle sign of $v$.
As described in \cref{L:sign for graph-like nbhds intersection}, this sign function makes the bijection between $\f_t(W) \times_{\interval^n} \f_{-t}(V)$ and $S$ sign-preserving.

As we now have a signed bijection of $\f_t(W) \times_{\interval^n} \f_{-t}(V)$ and $S$, comparing with the Serre diagonal in \cref{P:diagonal in terms of vertices} gives
\begin{align*}
	\cI (\f_t(W) \times_{\interval^n} \f_{-t}(V)) \left([\underline{0}, \underline{1}]^{\otimes n}\right) = &
	\sum_{v \in \vertices(\interval^n)} \sh(v) \cdot \cI(\f_t(W))(v^-) \cdot \cI(\f_{-t}(V))(v^+) \\ = & \
	\cI(\f_t(W)) \sms \cI(\f_{-t}(V)) \left([\underline{0}, \underline{1}]^{\otimes n}\right)
\end{align*}
for $t$ sufficiently large.

To prove (2), we first interchange the roles of $V$ and $W$ in (1) to obtain
\begin{equation*}
	\cI \big( \f_t(V) \times_M \f_{-t}(W) \big) =
	\cI \big( \f_t(V) \big) \sms \cI \big( \f_{-t}(W) \big).
\end{equation*}
But now $\f_t(V) \times_M \f_{-t}(W)=(-1)^{|V||W|}\f_{-t}(W) \times_M \f_{t}(V)$ by \cref{T:pull-back co-or}.
\hfill\qedsymbol