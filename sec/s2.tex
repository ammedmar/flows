% !TEX root = ../flows.tex

\section{Cubical topology}\label{S:cubical topology}

\subsection{Cubical complexes}\label{cubicalcomplexes}

The interpolation we develop between combinatorial and smooth topology proceeds through a cubulation of a manifold -- that is, a cubical complex homeomorphic to the manifold.
Such a structure is less common than that of a triangulation, so we present basic definitions, in a form best suited to our applications.

For simplicial complexes, vertices can always be given a partial order that restricts to a total order on each simplex, providing a way to identify each simplex with the standard simplex.
Furthermore, when two simplices meet along a common face, the induced ordering data for that face is consistent.
Categorically, such data is reflected in the fact that every simplicial complex is the realization of some simplicial set.
There is a parallel to this in the cubical setting, namely data required to compatibly identify each $n$-cube of a cubulation with the standard $n$-cube.

We thus begin with a formulation of cubical complexes containing such extra ordering data, as well as a description of the key features of cubical structures that will be needed for the analysis of our vector field flows in \cref{S:logistic}.

The \textbf{standard $n$-cube} is the subset of $\R^n$ defined by
\begin{equation*}
	\interval^n = \big\{ (x_1, \dots, x_n) \in \R^n\ |\ 0 \leq x_i \leq 1 \big\},
\end{equation*}
with the standard topology but with our preferred metric being the $L^\infty$ metric.
Denote $\{1, \dots, n\}$ by $\overline{n}$.
A partition $F = (F_0, F_{01}, F_1)$ of $\overline n$ defines a \textbf{face} of $\interval^n$ given by
\begin{equation*}
	\{(x_1, \dots, x_n) \in \interval^n\ |\ \forall \varepsilon \in \{0, 1\},\ i \in F_\varepsilon \Rightarrow x_i = \varepsilon\}.
\end{equation*}

We abuse notation and use the same notation for the partition and its associated face, referring to coordinates $x_i$ with $i \in F_{01}$ as \textbf{free}
and to the others as \textbf{bound}.
The \textbf{dimension} of $F$ is its number of free coordinates, and as usual the faces of dimension $0$ and $1$ are called vertices and edges, respectively.
The set of vertices of $\interval^n$ is denoted by $\vertices(\interval^n)$.

If $F_1 = \emptyset$, then we say that $F$ is an \textbf{initial} face; if $F_0 = \emptyset$, then we say that $F$ is a \textbf{terminal} face.
Let $\init_k(\interval^n)$ be the union of initial faces of dimension $k$ and $\term_k(\interval^n)$ the union of terminal faces of dimension $k$.

\begin{figure}[!h]
	\begin{subfigure}[b]{0.3\textwidth}
		\centering
		\includegraphics[scale=.7]{figures/cube.pdf}
	\end{subfigure}
	\hfill
	\begin{subfigure}[b]{0.3\textwidth}
		\centering
		\includegraphics[scale=.7]{figures/initial_final.pdf}
	\end{subfigure}
	\hfill
	\begin{subfigure}[b]{0.3\textwidth}
		\centering
		\includegraphics[scale=.7]{figures/l_infty_nbhd.pdf}
	\end{subfigure}
	\caption{On the left we have a representation of the standard $3$-cube $\interval^3$.
		In the center we depict $\init_1(\interval^3)$, the initial 1-dimensional faces, in blue and $\term_1(\interval^3)$, the terminal 1-dimensional faces, in red. On the right we depict an $\epsilon$-neighborhood of $F = (\{2\}, \{3\}, \{1\})$ in the $L^\infty$ metric.}
\end{figure}

The maps $\delta_i^\varepsilon \colon \interval^{n-1} \to \interval^{n}$ are defined for $\varepsilon \in \{0, 1\}$ and $i \in \overline{n}$ by
\begin{align*}
	\delta_i^\varepsilon(x_1, \dots, x_{n-1}) & = (x_1, \dots, x_{i-1}, \varepsilon, x_i, \dots, x_{n-1}),
\end{align*}
and any composition of these is referred to as a \textbf{face inclusion map}.

For $v \in \vertices(\interval^n)$ all coordinates are bound -- that is, $v_{01} = \emptyset$. Thus
$v$ is determined by the partition of $\overline n$ into $v_0$ and $v_1$, so
we have a bijection from the set of vertices of $\interval^n$ to the power set $\mathcal P(\overline n)$ of $\overline n$, sending $v$ to $v_1$.
The inclusion relation in the power set induces a poset structure on $\vertices(\interval^n)$ given explicitly by
\begin{equation*}
	v = (\epsilon_1, \dots, \epsilon_n) \leq w = (\eta_1, \dots, \eta_n) \iff \forall i,\ \epsilon_i \leq \eta_i.
\end{equation*}
We will freely use the identification of these posets.
The smallest and largest elements in $\mathcal P(\overline{n})$, which we denote $\underline{0}$ and $\underline{1}$, are the
initial and terminal vertices. Face embedding maps induce order-preserving maps at the level of vertices.

An \textbf{interval subposet} of $\mathcal P(\overline n)$ is one of the form $[v, w] = \{u \in \mathcal P(\overline n)\ |\ v \leq u \leq w\}$ for a pair of vertices $v \leq w$. There is a canonical bijection between faces of $\interval^n$ and such subposets, associating to $[v, w]$ the face $F$ defined by $F_\varepsilon = \{i \in \overline{n}\ |\ v_i = w_i = \varepsilon\}$ for $\varepsilon \in \{0, 1\}$.

The posets $\{\mathcal P(\overline n)\}_{n \geq 1}$ play the role for cubical complexes that finite totally ordered sets play for simplicial complexes.
Recall for comparison that one definition of an abstract ordered simplicial complex is as a pair $(V, X)$, where $V$ is a poset and $X$ is a collection of subsets of $V$, each with an induced total order, such that all singletons are in $X$ and subsets of sets in $X$ are also in $X$.
We have the following cubical analogue.

\begin{definition}\label{D:cubical}
	A \textbf{cubical complex} $X$ is a collection $\{ \sigma \}$ of finite non-empty subsets of a poset
	$\vertices(X)$, together with, for each $\sigma \in X$, an order-preserving bijection $\iota_\sigma \colon \sigma \to \mathcal P(\overline n)$ for some $n$, such that:
	\begin{enumerate}
		\item For all $v \in \vertices(X)$, $\{v\} \in X$,
		\item For all $\sigma \in X$ and all $[u,w] \subset \mathcal P(\overline n)$ the set $\rho = \iota_\sigma^{-1}([u,w]) \in X$ and the following commutes
		\begin{equation*}
			\begin{tikzcd} [row sep = tiny, column sep= small]
				\sigma \arrow[rr, "\iota_\sigma"] && \mathcal P(\overline n) \\
				& [-5pt] {[}u,w{]} \arrow[ur, hook] & \\
				\rho \arrow[uu, hook] \arrow[rr, "\iota_\rho"'] && \mathcal P(\overline m). \arrow[ul, "\cong"'] \arrow[uu, dashed]
			\end{tikzcd}
		\end{equation*}
	\end{enumerate}
	We refer to an element $\sigma \in X$ as a \textbf{cube} of $X$, refer to $\iota_\sigma \colon \sigma \to \mathcal P(\overline{n})$ as its \textbf{characteristic map},
	and refer to $n$ as its \textbf{dimension}. If $\rho \subseteq \sigma \in X$, we say that $\rho$ is a \textbf{face} of $\sigma$ in $X$.
	We identify elements in $\vertices(X)$ with the singleton subsets in $X$, referring to them as vertices.
\end{definition}

In analogy with the usual terminology in the simplicial setting, one could call these ``ordered cubical complexes," but we only work with these and have seen little use elsewhere for the unordered version.

Consider the category defined by the inclusion poset of a cubical complex $X$ and the subcategory ${\tt Cube}$ of the category ${\tt Top}$ of topological spaces whose objects are the $n$-cubes, identified with $\interval^n$, and whose morphisms are face inclusions.
The characteristic maps of $X$ define a functor from its poset category to $\mathtt{Cube}$, and we define its \textbf{geometric realization} as the colimit of this functor.
A \textbf{cubical structure} or \textbf{cubulation} on a space $S$ is a homeomorphism $h \colon |X| \to S$ from the geometric realization of a cubical complex.
We abuse notation and write $h \circ \iota_{|\sigma|}$ simply as $\iota_\sigma$ for any $\sigma \in X$ when a cubical structure $h \colon |X| \to S$ is understood.

Our definition sits between cubical sets \cite{jardine2002cubical} and cellular subsets of the cubical lattice of $\R^\infty$ \cite{kaczynski2006computational},
analogously to the way that abstract ordered simplicial complexes sit between simplicial sets and simplicial complexes.
The geometric realization construction makes our definition and the cubical lattice definition essentially equivalent.
Just as is the case for simplicial complexes, faces in cubical complexes are completely determined by their vertices.

\begin{figure}[h]
	\begin{subfigure}{.4\textwidth}
		\centering
		\includegraphics{figures/torus1.pdf}
		\caption{\textbf{Not} a cubulation of the torus}
	\end{subfigure}\qquad
	\begin{subfigure}{.4\textwidth}
		\centering
		\includegraphics{figures/torus2.pdf}
		\caption{A cubulation of the torus}
	\end{subfigure}
	\caption{The first cellular decomposition of a torus pictured above does not represent the geometric realization of a cubical complex, as each square has the same set of vertices. On the right, each square can be coherently identified with the standard square with initial vertex in the lower left corner and final vertex in the upper right corner. Therefore, (B) depicts a cubical structure on the torus.}
	\label{F:cubical structure}
\end{figure}

The vector field flow we define on cubulated manifolds in \cref{S:logistic} can be viewed as a smooth extension of the cubical poset structure.
We refine our description of this poset structure through identifying ``previous'' and ``next'' faces in a cube.

\begin{definition}\label{D:F decomposition}
	Let $F = (F_0, F_{01}, F_1)$ be a face of $\interval^n$. The $F$-\textbf{decomposition} of $\interval^n$ is the isomorphism $\interval^n \cong F^- \times F \times F^+$ where $F^- = (F_0 \cup F_{01}, F_1, \emptyset)$ and $F^+ = (\emptyset, F_0, F_1 \cup F_{01})$.
\end{definition}

An alternate definition of $F^+$ is as the face whose initial vertex is the terminal vertex of $F$ and whose terminal vertex is $\underline{1}$,
the terminal vertex of $\interval^n$.
Similarly, $F^-$ is the face whose terminal vertex is the initial vertex of $F$ and whose initial vertex is $\underline{0}$, the initial vertex of $\interval^n$. See \cref{F:decomposition}.

\begin{figure}[h!]
	\begin{subfigure}[b]{0.35\textwidth}
		\centering
		\includegraphics[scale=.7]{figures/decomposition.pdf}
	\end{subfigure}
	\begin{subfigure}[b]{0.35\textwidth}
		\centering
		\includegraphics[scale=.7]{figures/decomposition2.pdf}
	\end{subfigure}
	\caption{Examples of $F$-decompositions with $F = (\{2\}, \{3\}, \{1\})$ on the left and $F = (\{1,2\}, \emptyset, \{3\})$ on the right.}
	\label{F:decomposition}
\end{figure}

The special case of $F$-decompositions in which $F = v$, a vertex, merits its own consideration.

\begin{definition}\label{D:reciprocal}
	An ordered pair of faces $(F,F^\prime)$ of $\interval^n$ is said to be \textbf{reciprocal} if there exists a vertex $v$ such that $F = v^-$ and $F^\prime = v^+$. Equivalently, $(F, F')$ is reciprocal if and only if $F$ is initial and $F^\prime = F^+$, or if and only if $F'$ is terminal and $F = (F^{\prime})^-$.
\end{definition}

Consider the ordered set $\{\e_1, \dots, \e_n\}$ where $\e_i = \frac{\partial\ }{\partial x_i}$.
For any face $F$ of $\interval^n$, the ordered subset $\beta_F = \{\e_i\ |\ i \in F_{01}\}$ defines the \textbf{canonical orientation} of $F$.
We define the \textbf{shuffle sign} of $F$, denoted by $\sh(F) \in \{\pm 1\}$, to be $+1$ if the $F$-decomposition isomorphism is orientation preserving and $-1$ if not.
More explicitly, $\sh(F) = +1$ if the concatenation of the ordered sets $\beta_{F^-}$, $\beta_{F}$, and $\beta_{F^+}$ represents the same orientation as $\{\e_1, \dots, \e_n\}$, and $\sh(F) = -1$ otherwise.
This sign plays a key role in our applications, since we work over the ring of integers and this sign occurs in comparing products.


In the smooth setting, it will be convenient to extend characteristic maps in a cubulation.

\begin{lemma}\label{L:charextension}
	Let $i : \interval^m \to M$ be the characteristic map of a smooth cubical structure, and let $z \in \partial \interval^m$.  Then
	there is a neighborhood $U$ of $z$ in $\R^m$ and a smooth embedding $i_U: U \to M$ whose restriction to $U \cap \interval^m$ is $i$.
\end{lemma}

\begin{proof}
	By definition of smoothness, we can extend $i$ to a neighborhood of $z$ in $\R^n$, and by the continuity of the derivative there is a neighborhood on which this extension remains an immersion and so a possibly even smaller neighborhood on which it is an embedding.
\end{proof}


\subsection{Cubical cochains}

We can also define an ``algebraic realization" for a cubical complex in analogy to its geometric realization.
Let $\chains(\interval^1)$ be the usual cellular chain complex of the interval with integral coefficients.
Explicitly, $\chain0(\interval^1)$ is generated by the vertices $[\underline{0}]$ and $[\underline{1}]$, and $\chain1(\interval^1)$ is generated by the unique 1-dimensional face, denoted $[\underline{0},\underline{1}]$ in the interval subposet notation.
The boundary map is $\partial [\underline{0},\underline{1}]=[\underline{1}]-[\underline{0}]$.

Let $\chains(\interval^n) = \chains(\interval^1)^{\otimes n}$, with differential defined by the graded Leibniz rule.
Given a face inclusion $\delta_i^{\varepsilon} \colon \interval^n \to \interval^{n+1}$ the natural chain map $\chains(\delta_i^{\varepsilon}) \colon \chains(\interval^1)^{\otimes n} \to \chains(\interval^1)^{\otimes n+1}$ is defined on basis elements by
\begin{equation*}
	x_1 \otimes \cdots \otimes x_n \mapsto
	x_1 \otimes \cdots \otimes [\underline{\varepsilon}] \otimes \cdots \otimes x_n.
\end{equation*}
Regarding a cubical complex $X$ as a functor to $\mathtt{Cube}$, we can compose it with the chain functor above to obtain a functor to chain complexes.
The complex of \textbf{cubical chains} of $X$, denoted $C_*(X)$, is defined to be the colimit of this composition.
As one would expect, in each degree it is a free abelian group generated by the cubes of that dimension, and its boundary homomorphism sends the
generator associated to a cube to a sum of generators associated to its codimension-one faces with appropriate signs.

The \textbf{cubical cochains} of $X$ (with $\Z$ coefficients) is the chain complex $\cochains(X) = \Hom_\Z(\chains(X), \Z)$.
By abuse, we use the same notation and terminology for an element in $X$, its geometric realization in $|X|$,
and the corresponding basis elements in $\chains(X)$ and $\cochains(X)$.

We next recall the \textbf{Serre diagonal}.
Let $\Delta \colon \chains(\interval^1) \to \chains(\interval^1)^{\otimes 2}$ be defined on basis elements by
\begin{gather*}
	\Delta([\underline{0}]) = [\underline{0}] \otimes [\underline{0}], \qquad
	\Delta([\underline{1}]) = [\underline{1}] \otimes [\underline{1}], \qquad
	\Delta([\underline{0}, \underline{1}]) = [\underline{0}] \otimes [\underline{0}, \underline{1}] + [\underline{0}, \underline{1}] \otimes [\underline{1}].
\end{gather*}
Then, let
$\Delta \colon \chains(\interval^n) \to \chains(\interval^n)^{\otimes 2}$
be the composite
\begin{equation*}
	\begin{tikzcd}
		\chains(\interval)^{\otimes n} \arrow[r, "\Delta^{\otimes n}"] & \left( \chains(\interval)^{\otimes 2} \right)^{\otimes n} \arrow[r, "sh"] & \left( \chains(\interval)^{\otimes n} \right)^{\otimes 2},
	\end{tikzcd}
\end{equation*}
where $sh$ is the shuffle map that reorders tensor factors so that those in odd positions occur first. More explicitly, using Sweedler's notation,
if $x_i^{(1)}$ and $x_i^{(2)}$ are defined through the identity
\begin{equation*}
	\Delta(x_i) = \sum x_i^{(1)} \otimes x_i^{(2)},
\end{equation*}
then
\begin{equation}\label{E:Delta}
	\Delta (x_1 \otimes \cdots \otimes x_n) =
	\sum \pm \left( x_1^{(1)} \otimes \cdots \otimes x_n^{(1)} \right) \otimes
	\left( x_1^{(2)} \otimes \cdots \otimes x_n^{(2)} \right),
\end{equation}
where the sign is determined by the Koszul convention.

The \textbf{cup product} of cochains $\alpha, \beta \in \cochains(X)$ is defined using the Serre diagonal as follows\footnote{
	We follow the convention for evaluation of tensor products of cochains on tensor products of chains given by $(\alpha\otimes \beta)(x\otimes y)=\alpha(x)\beta(y)$. This convention is used for defining the cup product, for example, by Munkres \cite[Section 60]{Munk84}, Hatcher \cite[Section 3.2]{Hat02}, and Spanier \cite[Section 5.6]{Spa81}. But it disagrees with the conventions in Dold \cite[Section VII.7]{Dol72}, where there is a sign coming from the Koszul convention.}:
\begin{equation*}
	(\alpha \sms \beta)(c) = (\alpha \otimes \beta)\Delta(c).
\end{equation*}

We will use the following more explicit description of Serre's diagonal.

\begin{proposition}\label{P:diagonal in terms of vertices}
	The map $\Delta \colon \chains(\interval^n) \to \chains(\interval^n)^{\otimes 2}$ satisfies
	\begin{equation*}
		\Delta \big([\underline 0, \underline 1]^{\otimes n}\big) \ =
		\sum_{v \in \vertices(\interval^n)} \sh(v) \cdot v^- \otimes v^+.
	\end{equation*}
\end{proposition}

\begin{proof}
	In expression \eqref{E:Delta} each $x_i^{(1)}$ must be $[\underline 0]$ or $[\underline 0, \underline 1]$ and each $x_i^{(2)}$ must be $[\underline 0, \underline 1]$ or $[\underline 1]$.
	Moreover, if $x_i^{(1)} = [\underline 0]$ then $x_i^{(2)} = [\underline 0, \underline 1]$, and if $x_i^{(1)} = [\underline 0, \underline 1]$ then $x_i^{(2)} = [\underline 1]$.
	Hence, in each summand of \eqref{E:Delta}, the first and second tensor factors are reciprocal faces of $\interval^n$.
	Conversely, each vertex of $\interval^n$ determines such a summand.
	The proposition now follows from the identification of the shuffle sign with the sign arising from applying the Leibniz rule.
\end{proof}