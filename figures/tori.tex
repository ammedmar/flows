\begin{figure}
	\newcommand*{\xMin}{0}%
	\newcommand*{\xMax}{4}%
	\newcommand*{\yMin}{0}%
	\newcommand*{\yMax}{4}%
	\begin{subfigure}{.4\textwidth}
		\centering
		\begin{tikzpicture}[scale=.8]
		\draw[-{Latex[length=2mm]}] (-.5,\yMin)--(-.5,\yMax);
		\draw[-{Latex[length=2mm]}] (-.5,\yMin)--(-.5,\yMax-.5);
		\draw[-{Latex[length=2mm]}] (4.5,\yMin)--(4.5,\yMax);
		\draw[-{Latex[length=2mm]}] (4.5,\yMin)--(4.5,\yMax-.5);
		
		\draw[-{Latex[length=2mm]}] (\xMin, -.5)--(\xMax, -.5);
		\draw[-{Latex[length=2mm]}] (\xMin, 4.5)--(\xMax, 4.5);
		
		\draw [very thin,gray] (\xMin, \yMin) -- (\xMin, \yMax) -- (\xMax, \yMax) -- (\xMax, \yMin) -- (\xMin, \yMin);
		
		\draw [very thin,gray] (0.5*\xMax, \yMin) -- (0.5*\xMax, \yMax);
		
		\draw [very thin,gray] (\xMin, 0.5*\yMax) -- (\xMax, 0.5*\yMax);
		\end{tikzpicture}
		\caption{\textbf{Not} a cubulation}
	\end{subfigure}\qquad  
	\begin{subfigure}{.4\textwidth}
		\centering
		\begin{tikzpicture}[scale=.8]
		\draw[-{Latex[length=2mm]}] (-.5,\yMin)--(-.5,\yMax);
		\draw[-{Latex[length=2mm]}] (-.5,\yMin)--(-.5,\yMax-.5);
		\draw[-{Latex[length=2mm]}] (4.5,\yMin)--(4.5,\yMax);
		\draw[-{Latex[length=2mm]}] (4.5,\yMin)--(4.5,\yMax-.5);
		
		\draw[-{Latex[length=2mm]}] (\xMin, -.5)--(\xMax, -.5);
		\draw[-{Latex[length=2mm]}] (\xMin, 4.5)--(\xMax, 4.5);
		
		\foreach \i in {\xMin,...,\xMax} {
			\draw [very thin,gray] (\i,\yMin) -- (\i,\yMax);
		}
		\foreach \i in {\yMin,...,\yMax} {
			\draw [very thin,gray] (\xMin,\i) -- (\xMax,\i);
		}
		\end{tikzpicture}
		\caption{A cubulation of the torus}
	\end{subfigure}
	\caption{Faces in cubical complexes are completely determined by their vertices, implying that the first decomposition of a torus pictured above does not represent a cubical complex as each square has the same set of vertices. Orderings have been omitted from these representations.}
	\label{F: cubical structure}
\end{figure}