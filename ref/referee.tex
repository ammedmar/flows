\documentclass{amsart}
\input{../other/style}
\input{../other/usualcmds}
\addbibresource{../other/usualpapers.bib}

%%%%%%%%%%%%%%%%%%%%%%
% !TEX root = flows.tex

% colors
\definecolor{red}{rgb}{1,.5,.6}
\definecolor{blue}{rgb}{.5,.5,1}

% math commands
\newcommand{\R}{\mathbb{R}}              % reals
\newcommand{\Z}{\mathbb{Z}}              % integers
\newcommand{\N}{\mathbb{N}}              % natural
\newcommand{\norm}[1]{\Vert{#1}\Vert}    % norm
\newcommand{\id}{\mathrm{id}}            % identity
\newcommand{\e}{\mathbf{e}}              % canonical v-field
\newcommand{\f}{\mathbf{f}}              % logistic v-field
\newcommand{\interval}{\mathbb{I}}       % geometric interval
\newcommand{\chains}{C_{\ast}}           % chains
\newcommand{\cochains}{C^{\ast}}         % cochains
\newcommand{\chain}[1]{C_{#1}}           % chains one degree
\newcommand{\cochain}[1]{C^{#1}}         % cochains one degree
\newcommand{\sh}{\mathfrak{sh}}          % shuffle sign
\newcommand{\Or}{{\rm Det}}              % orientation bundle
\newcommand{\cman}{\mathrm{cMan}}        % c-manifolds
\newcommand{\cI}{\mathcal{I}}            % comparison map
\newcommand{\Hom}{\textup{Hom}}          % Hom complex
\newcommand{\init}{{\rm Init}}           % initial faces
\newcommand{\term}{{\rm Term}}           % terminal faces
\newcommand{\vertices}{{\rm Vert}}       % vertices
\newcommand{\uW}{\underline{W}}          % geometric cochain W
\newcommand{\uV}{\underline{V}}          % geometric cochain V
\newcommand{\Ginit}{G^{\mathrm{init}}}   % initial graph-like nbhd
\newcommand{\Gterm}{G^{\mathrm{term}}}   % initial graph-like nbhd

% typing shortcuts
\newcommand{\into}{\hookrightarrow}
\newcommand{\xr}{\xrightarrow}
\newcommand{\sms}{\smallsmile}
\newcommand{\pf}{\pitchfork}
\renewcommand{\th}{^{\mathrm{th}}}

\newcommand{\anibal}[1]{\textcolor{brown}{*Anibal: #1}}
\newcommand{\dev}[1]{\textcolor{blue}{*Dev: #1}}
\newcommand{\red}[1]{\textcolor{red}{#1}} % add commands here
\addbibresource{../other/bibliography.bib} % add references here
\usepackage{enumitem}
\setlist{label=\arabic{enumi}.,itemsep=\medskipamount, left=0pt}

%%%%%%%%%%%%%%%%%%%%%%
\title[Referee reply]{REFEREE REPLY 1 \\ Flowing from intersection product to cup product}

\newcommand{\ar}{\medskip\noindent\textit{Reply}:\ }
\newcommand{\tbw}{\ar \hrulefill}
\newcommand{\cas}{\ar C.A.S.}
\renewcommand{\thesection}{\arabic{section}}

\begin{document}
	\noindent\today
	\maketitle

	We would like to thank the reviewer for a careful and insightful analysis of our paper, and for the many suggestions improving its presentation.
	We copy their report for completeness.

	\section{Reviewer's summary}

	\noindent This paper addresses the compatibility of cup product and intersection product on the cochain level. An explicit flow is used to compare the products and show that they agree after flowing. Although this compatibility is assumed in the literature as folklore, this is the first rigorous account with a proof. The results and the ideas of the paper are important and could have a significant impact in the near future. The setting is at the moment restricted to cubical chains and associative products, but the authors promise further investigation of possible generalizations. Notice that the foundational paper on geometric homology by the same authors has not been published yet, as far as I know. That paper provides strong motivations and its results are also recalled in section 3. Nevertheless, the main theorem of the paper under review does not depend on the foundational paper as far as I can tell, with the exception of section 3. The paper is generally well written. The proofs are convincing. It would be helpful for the reader if the authors spelled out the ideas before diving into some technical proofs. I have some minor remarks about the exposition, the mathematics, and typos. I will be happy to recommend the paper for publication after these are addressed.

	\section{Reviewer's individual items}

	\begin{enumerate}
		\item I have the impression that the case of geometric cochains of negative codimension is excluded (this is reasonable in view of the fact that they represent homology). This should be stated to avoid pathological cases.

		\tbw

		\item In Figure 1(B), the lower W should be V.

		\ar Changed as suggested (C.A.S.).

		\item The main Theorem 1 has two statements. It might be preferable to have just one statement as the second is just a corollary with an extra sign coming from orientations.

		\cas

		\item Definition 6 defines the notion of a smooth map. Shortly after that, the notion of a weakly smooth map is just recalled as another notion due to Joyce, and it is remarked that it will not be required. However, later in the paper, the word 'weakly smooth' appears, for example in Theorem 11 and Definition 14. Can the authors clarify this?

		\tbw

		\item In the informal definition of transversality before Definition 10, one should stress that one needs a common image point.

		\cas

		\item In Definition 10, it is stated that transversality as defined here is equivalent to transversality of composition with maps from the iterated boundaries in the standard sense. But these are themselves manifolds with corners, so what does 'standard' mean? Is this a recursive definition?

		\tbw

		\item Definition 16: in the definition of small rank, does full rank mean rank equal to the dimension of the open stratum?

		\tbw

		\item As said above, the long paper [FMS22] is cited many times in section 3 and elsewhere, for example in Proposition 18, Theorems 22, 27, 28, 31. It would be helpful to give the exact reference for cited results.

		\ar Thank you for this suggestion. We certainly agree. Unfortunately, said monograph has not been published yet, so its internal numbering will most certainly change.

		\item Paragraph after Theorem 20: what is the domain and codomain of f?

		\cas

		\item Line -2 on page 20: 'lower' should be 'upper'.

		\cas

		\item In Lemma 39, observe that $(F_{01})^+ = F_0$ (from the definition).

		\cas

		\item At the end of the proof, the limit to infinity needs to be squared.

		\cas

		\item Lemma 40: v is the terminal vertex of F, not the initial. In the proof, epsilon should be u.

		\cas

		\item After the diagram on page 24: 'finite chain complex' should be 'finitely generated chain complex'.

		\cas

		\item In Definition 46, N should be M.

		\cas

		\item In Theorem 47, state that X is a cubulation of M.

		\cas

		\item The proof of Theorem 47 is hard to follow: does the induction step go from i-1 to i?

		\tbw

		\item I was misled because F- should be F in “projects surjectively onto F” and “the projection of .. to F”. We can choose u in (0,1) by inductive hypothesis.

		\tbw

		\item Proof of Lemma 51: put the composition of $f_T$ and $r_W$ between brackets.

		\cas

		\item Why is the fiber product replaced by an intersection at the end?

		\tbw

		\item Definition 53 (3): give a precise definition of 'bounded away'.

		\tbw

		\item Shortly afterward: say that r and u exist because of (3).

		\tbw

		\item Lemma 54: specify that the pair F and F' is ordered. In the proof, $L_\epsilon$ should be $N_\epsilon$.

		\cas

		\item The “plane” mentioned looks 2-dimensional; that is not the general case.

		\tbw

		\item First formula on page 31 and later: put indices for $C$ and $C'$ depending on the indexing sets.

		\tbw

		\item In the formula for $S$, specify that $v$ is chosen so that $v^-$ and $v^+$ have the right dimensions.

		\tbw
	\end{enumerate}

	\section{Other changes}

	\begin{enumerate}
		\item Update AMM affiliation.

		\item Should we change the title to flowing from intersection to cup product?
	\end{enumerate}
\end{document}